% !TEX TS-program = xelatex
% !TEX encoding = UTF-8 Unicode
% !Mode:: "TeX:UTF-8"

\documentclass{resume}
\usepackage{zh_CN-Adobefonts_external} % Simplified Chinese Support using external fonts (./fonts/zh_CN-Adobe/)
% \usepackage{NotoSansSC_external}
% \usepackage{NotoSerifCJKsc_external}
% \usepackage{zh_CN-Adobefonts_internal} % Simplified Chinese Support using system fonts
\usepackage{linespacing_fix} % disable extra space before next section
\usepackage{cite}

\begin{document}
\pagenumbering{gobble} % suppress displaying page number

\name{黎泽仁}

\basicInfo{
  \email{kould2333@gmail.com} \textperiodcentered\
  \phone{(+86) 180-1190-6217} \textperiodcentered\
  \github{https://github.com/KKould}
}

\section{\faUsers\ 工作/项目经历}
\datedsubsection{\textbf{KipDB}}{2022年7月 -- 至今}
\role{Rust}{个人开源项目}
\begin{onehalfspacing}
轻量级LSM存储引擎: https://github.com/KipData/KipDB
\begin{itemize}
  \item 支持嵌入式/单机存储/远程调用等多应用场景
  \item 参考LevelDB实现Leveled Compaction
  \item BenchMark写入吞吐量约为Sled的两倍
  \item 实现MVCC以支持ACID
\end{itemize}
\end{onehalfspacing}

\datedsubsection{\textbf{KipSQL}}{2023年6月 -- 至今}
\role{Rust}{合作开源项目}
\begin{onehalfspacing}
  基于KipDB持久化的轻量级SQL计算引擎: https://github.com/KipData/KipSQL
  \begin{itemize}
    \item 支持DML/DQL/DDL等基本CRUD
    \item 支持TPC-H Q1相关语法以及Join
    \item 参考TiDB的TableCodec设计,以KV数据库作为数据存储
    \item 目前支持:KipDB, Memory两种持久化方式. 可轻易拓展满足 get/set/remove/iter 基础api的KV数据库
    \item 实现RBO优化器及基本的SPJ优化: ColumnPruning / PushDownPredicates ..
  \end{itemize}
\end{onehalfspacing}

\datedsubsection{\textbf{Gitee} 深圳}{2023年2月 -- 2023年7月}
Ruby后端开发
\begin{itemize}
  \item CommitMessage支持MarkDown
  \item 重构WebHook
  \item 专业版Logo可配置化
\end{itemize}

% Reference Test
%\datedsubsection{\textbf{Paper Title\cite{zaharia2012resilient}}}{May. 2015}
%An xxx optimized for xxx\cite{verma2015large}
%\begin{itemize}
%  \item main contribution
%\end{itemize}

\section{\faCogs\ IT 技能}
% increase linespacing [parsep=0.5ex]
\begin{itemize}[parsep=0.5ex]
  \item 编程语言: Rust => Java > Go
  \item 平台: Window, Linux
  \item 开发: Docker
\end{itemize}

\section{\faGraduationCap\  教育背景}
\datedsubsection{\textbf{广东机电职业技术学院}, 广东, 广州}{2019 -- 2022}
\textit{专科}\ 软件技术

\section{\faInfo\ 其他}
% increase linespacing [parsep=0.5ex]
\begin{itemize}[parsep=0.5ex]
  \item 狂热开源爱好者, 维护有多个开源项目同时为RuskDesk、skip-list等热门仓库做出贡献
  \item 数据库爱好者,受TIDB启发从而开始学习Rust、LSM、SQL、Raft等
  \item 语言: 日语 - N3
\end{itemize}

%% Reference
%\newpage
%\bibliographystyle{IEEETran}
%\bibliography{mycite}
\end{document}
